\documentclass{article}
\usepackage{authblk,graphicx}
\usepackage{fancybox}% Enable the fancybox macro for landscape tables
\usepackage{multirow}
\usepackage{url}

%% Caption command good for all figures, indecies, etc.
\newcommand{\mycaption}[3]{
\caption[#1]{\textsf{#2.} \label{#1} \small{\linespread{1}#3}}
}

% Draft stuff
%
%********
\def\draftversion{Y}                % Y for draft, N for final version
%********
\def\note[#1]#2{\message{(#1)}\if\draftversion
{\noindent\em[#2]\/}\fi}
%
% Mark all printed pages for draft version using dvips \specials
%
\if \draftversion Y
\special{!userdict begin /bop-hook{gsave 200 30 translate 66 rotate
  /Times-Roman findfont 160 scalefont setfont 0 0 moveto 0.85 setgray
  (Preliminary) show grestore}def end}
\fi
\def\@oddhead{
         \vbox{\hbox to\hsize{\sl \rightmark \hfill  }
              \vspace{2pt} \hbox to\hsize{\hrulefill}}}


\begin{document}
\title{ANITA-4 AWARE Installation Instuctions}
\author{R.J. Nichol}
  
\date{\today}
\maketitle

%%\tableofcontents 

\section{Introduction}
During the flight it is critical that we are able to monitor the health of the instrument. This is particularly true during the line-of-sight period in which we obtain the majority of our telemetry data. It is critical that we use this time to ensure that the instrument is working as well as it can.

I would suggest people read Marty's data transfer website which describes how LOS (line-of-sight) data arrives at the los machines in the hangar and how TDRSS and Openport data arrive on the Tdrss1 machine in the OCC in Palestine\cite{web:martyDataTransfer}.

Once the data has arrived on the acquisition machines it gets shipped around the world. Most places use Marty's fantastic data transfer programs, anita-aware and UCL use some rysnc scripts.

A suite of database-based monitoring programs have been developed and are described elsewhere~\cite{web:anitaMonitoring}.

This document will describe how to install AWARE monitoring software and explain how to run the various scripts that make up AWARE at anita-aware.

\section{Installing AWARE on your GSE}
All of the ANITA offline (and quasi-online) software is now hosted on github. The approved method of installing AWARE is using {\tt anitaBuildTool}.

\subsection{Pre-requisites}
Before you start with the ANITA software a few pre-requisites need to be installed. Most of these you should be able to install from your preferred package manager or by downloading from the sites below.
\begin{enumerate}
\item ROOT -- http://root.cern.ch  -- version 5.34 and 6.06 are known to work with the ANITA software.
\item FFTW -- http://www.fftw.org
\item CMAKE -- https://cmake.org  (minimum version 2.8.10)
\end{enumerate}

\subsection{Installing AWARE}
The following set of commands can be used to install AWARE and the various ANITA libraries.
\begin{enumerate}
\item Set the ANITA\_UTIL\_INSTALL\_DIR environmental variable to point to the location where the anita libraries and binaries and calibration files will be installed. Automatic bin, include, lib, share subdirectories will be created.
\item Clone the anitaBuildTool repository \\{\tt git clone https://github.com/anitaNeutrino/anitaBuildTool.git}
\item Change to the directory \\ {\tt cd anitaBuildTool}
\item Actually download and install the software\\{\tt ./buildAnita.sh 3 0 1}\\  3 is the number of jobs (Cosmin recommends number of cores plus one), 0 is to suppress configuration and 1 is to build AWARE
\end{enumerate}
If everything works well the libraries and binaries of the ANITA software will be installed. The anitaBuildTool downloads all of the necessary libraries to run the offline software and AWARE.

\section{Setting up AWARE}
Once the installation is successfull a few more environmental variables are needed to run AWARE. An example of these are shown below from the setup on anitarf. These setup scripts are in the {\tt anitaBuildTool/components/anitaTelem} directory you should create one for your site.
\begin{verbatim}
  #All of these need to be set
export ANITA_BUILD_TOOL_DIR=/home/radio/anitaCode/anitaBuildTool
export ANITA_BASE_DIR=${ANITA_BASE_DIR}
export AWARE_BASE_DIR=${ANITA_BUILD_TOOL_DIR}/components/aware
export ANITA_AWARE_FILEMAKER_DIR=${ANITA_BUILD_TOOL_DIR}/components/anitaAwareFileMaker/
export ANITA_TELEM_DIR=${ANITA_BUILD_TOOL_DIR}/components/anitaTelem


#This should be ANITA_UTIL_INSTAll_DIR/bin
export ANITA_BIN_DIR=/usr/local/anita/bin/

#This is the output directory which will need to be mounted on the web server
export AWARE_OUTPUT_DIR=/nasc/aware/output/

#This is the input directory which telemetry things will be copied to
export ANITA_TELEM_DATA_DIR=/nasc/flight1617/telem/

#This is a local site specifc script that may or may not be needed
export AWARE_SITE_SCRIPT=${ANITA_TELEM_DIR}/startAnitaRFCopyScripts.sh

export PYTHONPATH=${AWARE_BASE_DIR}/python/:\
${ANITA_BUILD_TOOL_DIR}/components/pyinotify/python2
\end{verbatim}

\section{AWARE Directory Structures}
For simplicity there are two criticial environmentals that describe where the input data {\tt \$ANITA\_TELEM\_DATA\_DIR} and the AWARE output dir {\tt \$AWARE\_OUTPUT\_DIR} will be located.

\subsection{Input Directory Structures}
The {\tt \$ANITA\_TELEM\_DATA\_DIR} environmental variable points to the location where the input telemetry data is located. This directory will also host the pseudo-raw and root data that is created by processing the raw telemetry files. An example of the data structure is shown below. I am not entirely sure how closely this mirrors the setup of the GSE telemetry directories. In the worst case the directory structure required below could be formed by extensive use of symlinks.

\begin{verbatim}
/nasc/flight1617/telem/fast_tdrss  #The fast_tdrss numerical directories and files
/nasc/flight1617/telem/iridium  #The iridium numerical directories and files
/nasc/flight1617/telem/openport  #The openport numerical directories and files
/nasc/flight1617/telem/raw   #The pseudo raw flight data made by processTelemFile
/nasc/flight1617/telem/raw_los  #The raw_los numerical directories and files
/nasc/flight1617/telem/root #The root data made by checkIfRunNeedsProcessing
/nasc/flight1617/telem/slow_tdrss  #The slow_tdrss numerical directories and files
\end{verbatim}

\subsection{Output Data}
AWARE is a website so certain directories need to be mounted on the web server. The raw html, php and javascript of AWARE is located in the {\tt anitaBuildTool/components/aware/web} directory. This should be mounted on a web server in some way. The {\tt AWARE\_OUTPUT\_DIR} should point to a directory which is symlinked to {\tt anitaBuildTool/components/aware/web/output}. To setup AWARE for ANITA-4 one needs to run the setupAware.sh script in {\tt anitaBuildTool/components/aware/web} with ANITA4 as the argument.


\section{AWARE Scripts}
There are two steps to the AWARE processing.
\begin{itemize}
\item Converting telemetry packets into pseudo-raw data as would be found on the flight computer.
\item Processing the raw data into ROOT and JSON files which are used by MagicDisplay and AWARE, respectively.
\end{itemize}

\subsection{Pseudo-Raw Data Creation}
As each new telemetry file arrives on anita-aware it is automatically processed into pseudo-raw data. One 'interesting' feature is that the processing scripts need to determine the run number for each packet using solely the information in the packets. There are only a few packets that contain information about the run number,including {\it RunStart\_t}, {\it OtherMonitor\_t} and {\it LogWatchdStart\_t}. A consequence of this determination of run number using the telemetry data means that there are edge effects where packets can be assigned to the wrong run number.

The code and scripts for processing the telemetry data in to pseudo-raw data are all located in {\tt /home/anita/Code/anitaTelem}. The important scripts are:
\begin{description}
\item{setupAwareVariablesForTelem.sh} \\Sets the various environmental variables needed by the aware processing scripts.
\item{processTelemFile} \\Runs over one or more telemetry files and creates pseudo-raw files in the raw data directory, e.g  {\texttt /data/flight1617/telem/raw\_los/00278/000175}
\item{newTelemFileWatcher.py} \\A python script that uses libinotify to call processTelemFile when new telemetry files arrive on anita-aware.
\item{startAwareTelem.sh} \\A simple script which checks if the correct environmental variables are set, and checks if the processing and rootifying scripts are running.
\end{description}

\subsection{ROOT and JSON File Creation}
Once the pseudo-raw data has been created it is ready for conversion into ROOT and JSON files. The important scripts in this phase are:
\begin{description}
\item{rootAndJsonFileLoop.sh} A shell script that checks whether recent runs have any new data and if they do calls checkIfRunNeedsProcessing.py
\item{checkIfRunNeedsProcessing.py} A python script that creates or updates ROOT files when a pseudo-raw data directory has new data since the last ROOT file creation.
\end{description}

\subsubsection{ROOT File Creation}
In {\tt anitaTreeMaker} are the worker scripts and programs that process the raw data in ROOT data. These are almost identical to the scripts used to produce ROOT data files from the full raw data set. In most circumstances you will not need to run these scripts manually.

\subsubsection{JSON File Creation}
In {\tt anitaAwareFileMaker} are the worker scripts and programs that process ROOT data into JSON data which can be read by the AWARE web monitor. Again in most circumstance you will not need to run these scripts manually.

\section{AWARE Web pages}
The raw php, html and javascript files that make up AWARE are in {\tt /home/anita/Code/awareTelem}. This is the directory which is mounted on the web server at \url{http://157.132.95.98/awareTelem/}. In brief, the files {\tt web/config/ANITA4} are the ANITA4 customised files which make up the AWARE web monitor.

\section{Directory Structure}
The directory structure in {\tt /data/flight1617/telem/} is summarised below:
\begin{description}
\item{{\tt raw}} \\The pseudo-raw run directories are created here
\item{{\tt root}} \\The ROOT run directories are created here
\item{{\tt raw\_los}} \\Copied from los-acq
\item{{\tt fast\_tdrss}} \\Copied from tdrss1
\item{{\tt slow\_tdrss}} \\Copied from tdrss1
\item{{\tt openport}} \\Copied from tdrss1, unless openport is running in non-flight mode and copying directly here
\item{{\tt iridium}} \\Copied from tdrss1
\item{{\tt aware}} \\The AWARE directory
\item{{\tt aware/output}} \\The AWARE output directory.
\item{{\tt aware/output/ANITA4}} \\The AWARE output directory for ANITA-4.
\item{{\tt aware/output/ANITA4/aux}} \\Text files sent from the payload (except config files) end up here.
\item{{\tt aware/output/ANITA4/cmdEcho}}\\ Command echos from the payload
\item{{\tt aware/output/ANITA4/cmdSent}} \\Command logs from the two commanding machines
\item{{\tt aware/output/ANITA4/db}} \\The flat text database files like the runNumberMap and the lastOpenpportFile live here, these are all automatically created. Bad things may happen if you change them.
\item{{\tt aware/output/ANITA4/event}} \\Event files produced by processTelemFile
\item{{\tt aware/output/ANITA4/ghd}} \\The currently less than useful telemetry file information
\item{{\tt aware/output/ANITA4/log}} \\Logs from running the aware processing scripts
\item{{\tt aware/output/ANITA4/runXX}} \\The actual JSON data from each run, also linked into a date directory for searching by date.
\end{description}

\section{Command Copy Script}
The AWARE monitor uses the JSON files written by cmdTui to display the commands sent to the payload (for easy comparison with the command echos received from the payload). The command copying script is \\{\tt /home/anita/Code/anitaTelem/copyCmdSendFromTdrss.sh} and the json command log files will be written to {\tt \$AWARE\_OUTPUT\_DIR/ANITA4/cmdSend/palestine} and {\tt \$AWARE\_OUTPUT\_DIR/ANITA4/cmdSend/los}. Modified versions of these scripts will be needed for the sent command panel on AWARE to be populated.



\begin{thebibliography}{99}
        
\bibitem{web:anitaMonitoring}
 Delaware monitoring website, \url{http://www.bartol.udel.edu/gp/anita/}

\bibitem{web:martyDataTransfer}
  Marty's data transfer website, \url{https://physics.wustl.edu/marty/anita/data_xfer/how_it_works_14.html}

\end{thebibliography}

\end{document}

